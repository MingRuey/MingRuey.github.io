% # -*- coding:utf-8 -*-
%% start of file `resume.tex'.
%% Author imchou239@gmail.com 
%
% Copyright 2006-1008 Xavier Danaux (xdanaux@gmail.com).
% This work may be distributed and/or modified under the
% conditions of the LaTeX Project Public License version 1.3c,
% available at http://www.latex-project.org/lppl/.

\documentclass[11pt,a4paper]{moderncv}

\usepackage{fontspec}
\setmainfont{Roboto}
\usepackage[slantfont ,boldfont]{xeCJK} 
\usepackage{xcolor}

\setmainfont{Roboto}
\setCJKmainfont{Noto Sans CJK TC}
\setCJKsansfont{Noto Sans CJK TC}
\setCJKmonofont{Noto Sans CJK TC}
\XeTeXlinebreaklocale "zh"
\XeTeXlinebreakskip = 0pt plus 1pt minus 0.1pt

% moderncv themes
\moderncvtheme[blue]{classic}                 
% optional argument: 'blue' (default), 'orange', 'red', 'green', 'grey' and 'roman' (for roman fonts, instead of sans serif fonts)

% adjust the page margins
\usepackage[scale=0.9]{geometry}

% change the colmn width of dates
%\setlength{\hintscolumnwidth}{3cm}

% Classic theme only, change the width of name placeholder 
% (to leave more space for your address details
%\AtBeginDocument{\setlength{\maketitlenamewidth}{6cm}}
\AtBeginDocument{\recomputelengths}

% personal data
\firstname{周}
\familyname{明叡}

% optional
\title{Ming Ruey(Ray) Chou}    
  %\address{1990/09/11}{}
  %\fax{fax}
\mobile{0966525557}
\address{}{}
\email{imchou239@gmail.com}
  %\homepage{Blog: http://geekplux.com}
\social[github]{MingRuey}
\extrainfo{%
  Twitter: imchou239\\
  台北市伊通街125巷3號6樓
}
\photo[55pt]{RayModified_Cropped.jpg}
%\quote{They are just things that nobody can know. Your situation is just an accident of life.}

% Suppress automatic page numbering for CVs longer than one page
%\nopagenumbers{}


%----------------------------------------------------------------------------------
%            content
%----------------------------------------------------------------------------------
\begin{document}
\maketitle
\vspace*{-14mm}

\section{技能}
\cvline{\textbf{程式語言}}{Python:\space\space\footnotesize{Linux - IDE \& Virtualenv/Docker, \space\space Windows - Anaconda},\quad\normalsize Java,\quad IDL, \quad Labview}
\cvline{\textbf{機器學習}}{TensorFlow,\quad Keras, \quad Scikit-Learn, \quad Pandas,\quad LightGBM 等主流套件}
\cvline{\textbf{資料視覺化}}{Matplotlib,\quad Seaborn;\quad OriginLab}
\cvline{\textbf{資料庫}}{SQL \& MySQL}
\cvline{\textbf{語言能力}}{中文:\space\space 母語;\quad 英文:\space\space 流利}
\vspace{-0.6\baselineskip}

\section{實作經驗}

\subsection{\href{https://www.kaggle.com/c/imaterialist-challenge-fashion-2018}{Kaggle - iMaterialist Challenge(Fashion) at FGVC5}}
\cvline{\vspace{-1cm}\scriptsize{\hfill 圖片多標籤分類} \newline \footnotesize{Top 16\%} }
{\footnotesize{$\bullet$ 編寫多執行緒的腳本加速下載超過百萬張訓練集圖片\space\space 
$\bullet$以 Keras 對預訓練模型進行遷移式學習 \newline $\bullet$ 使用 基於頻譜的顯著性分析 與 GrabCut 等前背景分離技術進行預處理。}}

\subsection{\href{https://www.kaggle.com/c/avito-demand-prediction}{Kaggle - Avito Demand Prediction Challenge}}
\cvline{\vspace{-1cm}\scriptsize{\hfill 商品成交率預測} \newline \footnotesize{Top 27\%} }
{\footnotesize{$\bullet$ 以 Pandas進行補值、標準化、One-Hot Encoding 等資料前處理\space\space $\bullet$ 使用梯度提升決策樹 GBDT(LightGBM) 
\newline $\bullet$ 反覆進行探索式資料分析 EDA、特徵選擇與特徵工程,提高模型預測率}}

\subsection{\href{https://www.kaggle.com/c/google-ai-open-images-object-detection-track}{Kaggle - Google AI Open Images - Object Detection Track}}
\cvline{\vspace{-1cm}\scriptsize{\hfill 影像物件偵測} \newline \footnotesize{Top 19\%}}
{\footnotesize{$\bullet$ 訓練集為目前最大的物件偵測資料庫\href{https://storage.googleapis.com/openimages/web/index.html}{ Open Images Dataset V4}\space\space $\bullet$ 將資料封裝成高效能的TFRecord格式
\newline $\bullet$ 以\href{https://github.com/tensorflow/models/tree/master/research/object_detection}{TensorFlow的物件偵測 API} 對預訓練過的 Faster-RCNN 進行遷移式學習。}}

\subsection{\href{https://www.kaggle.com/c/rsna-pneumonia-detection-challenge}{Kaggle - RSNA Pneumonia Detection Challenge}}
\cvline{\vspace{-1cm}\scriptsize{\hfill X光影像肺炎偵測} \newline \footnotesize{Top 21\%}}
{\footnotesize{$\bullet$ 從無到有編寫了純Tensorflow的 Unet 及 Faster-RCNN 架構\space\space $\bullet$ 以 Tensorflow Estimator API 自動化訓練模型
\newline $\bullet$ 以對半分割、旋轉與翻轉做資料增強提高模型預測率}}

\subsection{\href{https://tbrain.trendmicro.com.tw/Competitions/Details/2}{T-Brain AI實戰吧- 台灣ETF價格預測競賽}}
\cvline{\scriptsize{\hfill 時間序列預測}}
{\footnotesize{採用ARIMA對時間序列進行預測,也嘗試把 RSI、隨機指標(\%K\%D)、威廉指標、Chaikin擺動指標、順勢指標等傳統的股市分析指數作為資料特徵,以隨機森林方法進行預測。}}
\vspace{-0.6\baselineskip}

\section{學歷}
% \cventry{year--year}{Job title}{Employer}{City}{}{Description}
% arguments 3 to 6 are optional
\cventry{\footnotesize{2013 - 2016}}{碩士}{國立台灣大學物理學研究所}{}{}{}
\cventry{\footnotesize{2009 - 2013}}{學士}{國立台灣大學物理學系}{}{}{}
\cventry{}{Coursera 線上課程}{}{}{}{}
\cvlistitem{Algorithms 4E, Robert Sedgewick and Kevin Wayne; Princeton}
\cvlistitem{機器學習基石, 林軒田; NTU}
\cvlistitem{Machine Learning, Andrew Ng; Standford}
\cvlistitem{Introduction to SQL, Charles Severance; Michigan}
\cvlistitem{SQL for Data Science, Sadie St. Lawrence; UC Davis}
\vspace{-0.6\baselineskip}

\section{論文}
\cvline{\footnotesize{碩士論文}}{\textbf{初探軟顆粒懸浮液流變學:自製流變儀}}{}
\cvline{}{\href{http://www.phys.sinica.edu.tw/jctsai/Ray2016/}{\emph{www.phys.sinica.edu.tw/jctsai/Ray2016/}}}
\vspace{-0.6\baselineskip}

\section{其他經歷}
\cventry{\footnotesize{2018}}{雙橡海外教育}{課程設計與專案教練}{\href{http://www.twinoaks-edu.com/}{www.twinoaks-edu.com/}}{}{}
\cventry{\footnotesize{2016 - 2017}}{教育替代役168梯次}{花蓮信義國小}{}{}{}


% \section{Computer skills}
% \cvcomputer{category 1}{XXX, YYY, ZZZ}{category 4}{XXX, YYY, ZZZ}
% \cvcomputer{category 2}{XXX, YYY, ZZZ}{category 5}{XXX, YYY, ZZZ}
% \cvcomputer{category 3}{XXX, YYY, ZZZ}{category 6}{XXX, YYY, ZZZ}

% \section{Interests}
% \cvline{橋牌}{\small 沒有高手的程度 也要有高手的風度}
% \cvline{桌遊}{\small 買了一屋子遊戲 找不到一桌朋友}
% \cvline{鋼琴}{\small 學音樂的小孩比較會敲鍵盤}

% \renewcommand{\listitemsymbol}{-} % change the symbol for lists

% \section{Extra 1}
% \cvlistitem{Item 1}
% \cvlistitem{Item 2}
% \cvlistitem[+]{Item 3}            

% \section{Extra 2}
% \cvlistdoubleitem[\Neutral]{Item 1}{Item 4}
% \cvlistdoubleitem[\Neutral]{Item 2}{Item 5}
% \cvlistdoubleitem[\Neutral]{Item 3}{}

%% Publications from a BibTeX file
% \nocite{*}
% \bibliographystyle{plain}
% \bibliography{publications}
% \begin{thebibliography}{}
% \bibitem[]{}
% \end{thebibliography}


\end{document}


%% end of file `resume.tex'.

%%% Local Variables:
%%% mode: latex
%%% TeX-command-extra-options: "-shell-escape"
%%% TeX-master: t
%%% TeX-engine: xetex
%%% End: